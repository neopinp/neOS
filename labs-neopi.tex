%%%%%%%%%%%%%%%%%%%%%%%%%%%%%%%%%%%%%%%%%
%
% CMPT 424N
% Some Semester
% Lab/Assignment/Project X
%
%%%%%%%%%%%%%%%%%%%%%%%%%%%%%%%%%%%%%%%%%

%%%%%%%%%%%%%%%%%%%%%%%%%%%%%%%%%%%%%%%%%
% Short Sectioned Assignment
% LaTeX Template
% Version 1.0 (5/5/12)
%
% This template has been downloaded from: http://www.LaTeXTemplates.com
% Original author: % Frits Wenneker (http://www.howtotex.com)
% License: CC BY-NC-SA 3.0 (http://creativecommons.org/licenses/by-nc-sa/3.0/)
% Modified by Alan G. Labouseur  - alan@labouseur.com
%
%%%%%%%%%%%%%%%%%%%%%%%%%%%%%%%%%%%%%%%%%

%----------------------------------------------------------------------------------------
%	PACKAGES AND OTHER DOCUMENT CONFIGURATIONS
%----------------------------------------------------------------------------------------

\documentclass[letterpaper, 10pt,DIV=13]{scrartcl} 

\usepackage[T1]{fontenc} % Use 8-bit encoding that has 256 glyphs
\usepackage[english]{babel} % English language/hyphenation
\usepackage{amsmath,amsfonts,amsthm,xfrac} % Math packages
\usepackage{sectsty} % Allows customizing section commands
\usepackage{graphicx}
\usepackage[lined,linesnumbered,commentsnumbered]{algorithm2e}
\usepackage{listings}
\usepackage{parskip}
\usepackage{lastpage}

\allsectionsfont{\normalfont\scshape} % Make all section titles in default font and small caps.

\usepackage{fancyhdr} % Custom headers and footers
\pagestyle{fancyplain} % Makes all pages in the document conform to the custom headers and footers

\fancyhead{} % No page header - if you want one, create it in the same way as the footers below
\fancyfoot[L]{} % Empty left footer
\fancyfoot[C]{} % Empty center footer
\fancyfoot[R]{page \thepage\ of \pageref{LastPage}} % Page numbering for right footer

\renewcommand{\headrulewidth}{0pt} % Remove header underlines
\renewcommand{\footrulewidth}{0pt} % Remove footer underlines
\setlength{\headheight}{13.6pt} % Customize the height of the header

\numberwithin{equation}{section} % Number equations within sections (i.e. 1.1, 1.2, 2.1, 2.2 instead of 1, 2, 3, 4)
\numberwithin{figure}{section} % Number figures within sections (i.e. 1.1, 1.2, 2.1, 2.2 instead of 1, 2, 3, 4)
\numberwithin{table}{section} % Number tables within sections (i.e. 1.1, 1.2, 2.1, 2.2 instead of 1, 2, 3, 4)

\setlength\parindent{0pt} % Removes all indentation from paragraphs.

\binoppenalty=3000
\relpenalty=3000

%----------------------------------------------------------------------------------------
%	TITLE SECTION
%----------------------------------------------------------------------------------------

\newcommand{\horrule}[1]{\rule{\linewidth}{#1}} % Create horizontal rule command with 1 argument of height

\title{	
   \normalfont \normalsize 
   \textsc{CMPT 424N - Fall 2024 - Dr. Labouseur} \\[10pt] % Header stuff.
   \horrule{0.5pt} \\[0.25cm] 	% Top horizontal rule
   \huge Labs  \\     	    % Assignment title
   \horrule{0.5pt} \\[0.25cm] 	% Bottom horizontal rule
}

\author{Neo Pi \\ \normalsize Neo.Pi1@Marist.edu}

\date{\normalsize\today} 	% Today's date.

\begin{document}
\maketitle % Print the title

%----------------------------------------------------------------------------------------
%   start PROBLEM ONE
%----------------------------------------------------------------------------------------
\section{Lab One}

\subsection{What are the advantages and disadvantages of using the same system call interface for manipulating both files and devices?}
The advantage of using the same system call interface for both files and devices is so that each device can be accessed as thought it was a file within the system. Since kernel already deals with devices through a file interface, adding code to support a different kind of file interface is possible. 
The disadvantage of using the same interface is that the functionality of some devices might not be able to be accessed. Loss of functionality or performance could be a result.

\subsection{Would it be possible for the user to develop a new command interpreter using the system call interface provide by the operating system? How?}
Yes, a user can develop a new command interpreter by using the system call interface provided by the operating system.
You can develop it by specifically using system calls to manage processes and file-related system calls for i/o operations. 

\pagebreak
\section{Lab Two}

\subsection{How is your console like the ancient TTY subsystem in Unix as described in https://www.linusaskesson/programming/tty/?}
Originally, the TTY system was designed for text-based input and output. Similarly, the current console we are working on operates the same way. It takes character-based commands, and handles commands by processing input as text characters. The input is buffered line-by-line, just as it is in our own OS. Anohter important factor, is that the TTV subsystem was the primary interface for the system, and similarly our OS provides a similar interactive interface. Lastly, the interaction is all in real time, when issuing commands and receiving feedback. 


\pagebreak
\section{Lab Three}
\subsection{Explain	the	difference	between	internal and external fragmentation.}
Internal fragmentation happens when memory is allocated in block larger than what is required. The memory that is not used inside the allocated block becomes useless if the process does not require the whole block. External fragmentation happens when there is enough free memory, but the free memory is scattered in different blocks. Even though there is enough free memory, there are no single blocks large enough for the request. 
Overall, internal fragmentation contains wasted space within memory blocks, while external fragmentation has wasted space between memory blocks. 
\subsection{Given five memory partitions of 100KB, 500KB, 200KB, 300KB, and 600LB (in that order), how would optimal, first-fit best-fit, and worst-fit algorithms place processes of 212KB, 417KB, 112KB, and 426KB (in that order)?}

For optimal placement, it would work like this: 
212KB : 300KB 
417KB : 500KB 
112KB : 200KB 
426KB : 600KB 

First-Fit: 
212KB : 500KB 
417KB : 600KB 
112KB : 200KB 
426KB : 

Best-Fit: 
212KB : 300KB, 
417KB : 500KB, 
112KB : 200KB, 
426KB : 600KB 

Worst-Fit: 
212KB : 600KB, 
417KB : 500KB, 
112KB : 300KB, 
426KB : 




%----------------------------------------------------------------------------------------
%   end PROBLEM Two
%----------------------------------------------------------------------------------------

\pagebreak

%----------------------------------------------------------------------------------------
%   start PROBLEM Three
%----------------------------------------------------------------------------------------
\section{Lab 4}
\subsection{What is the relationship between a guest operating system and a host operating system like VMware? What factors need to be considered in choosing the host operating system?}
The relationship involves how the host manages the hardware resources and virtualizes them for the guest OS. The guest OS runs as if it were operating directly on a physical machine, but it instead operates on virtual hardware. The guest OS relies on the host OS for certain things such as CPU and RAM. Some factors to be considered are: 
The compatibility with the virtualization software 
Performance
Resource Allocation 
Security 


%----------------------------------------------------------------------------------------
%   end PROBLEM THREE
%----------------------------------------------------------------------------------------

\pagebreak

%----------------------------------------------------------------------------------------
%   REFERENCES
%----------------------------------------------------------------------------------------
% The following two commands are all you need in the initial runs of your .tex file to
% produce the bibliography for the citations in your paper.
\bibliographystyle{abbrv}
\bibliography{lab01} 
% You must have a proper ".bib" file and remember to run:
% latex bibtex latex latex
% to resolve all references.

\pagebreak

%----------------------------------------------------------------------------------------
%   Appendix
%----------------------------------------------------------------------------------------

\section{Appendix}

\subsection{Some JavaScript source code listings}

\lstset{numbers=left, numberstyle=\tiny, stepnumber=1, numbersep=5pt, basicstyle=\footnotesize\ttfamily}
\begin{lstlisting}[frame=single, ]  
var A = [5,0,5,6,6,8,45,50];

function solve(A) {
    // Base case to stop the recursion.
    if (A.length == 1) {
        if (A[0] % 5 == 0) {
            var retVal = 1;
        } else {
            var retVal = 0;
        }
        return retVal;
    } else {
        // Divide.
        var divPoint = Math.floor( A.length / 2);
        var left  = A.slice(0, divPoint);
        var right = A.slice(divPoint, A.length);
        
        // Conquer.
        var left5s   = solve(left, level+1);
        var center5s = straddle(left, right);
        var right5s  = solve(right, level+1);             
        
        // Combine.
        return Math.max(left5s, Math.max(center5s, right5s));
    }
}

function straddle(left, right) {
    var retVal = 0;
    if ((left[left.length-1] % 5 == 0) && (right[0] % 5 == 0)) {
        // Count back the 5's on the left going from right to left.
        var leftCount = 0;
        var index = left.length-1;
        while ( (index >= 0) && (left[index] % 5 == 0) ) {
            index--;
            leftCount++;
        }
        // Count forward the 5's on the right going from left to right.
        var rightCount = 0;
        while ( (rightCount < right.length) && (right[rightCount] % 5 == 0) ) {
            rightCount++;
        }
        // Return the sum of the straddling 5s on the left and right.
        retVal = leftCount + rightCount;
    }
    return retVal;
}
\end{lstlisting}

\end{document}
